\specsection{ЗАКЛЮЧЕНИЕ \hspace{1.25cm}}
\vspace{\baselineskip}

В ходе выполнения работы были решены следующие задачи:

\begin{enumerate}[label=---]
	\item описан и реализован алгоритм полного перебора для решения задачи коммивояжера;
	\item описан и реализован муравьиный алгоритм для решения задачи коммивояжера;
	\item проведена параметризация муравьиного алгоритма на двух классах данных;
	\item проведен сравнительный анализ времени выполнения и трудоемкостей реализаций.
\end{enumerate}

Применимость алгоритмов зависит от того, насколько велик размер матрицы расстояний.
При размерах $N < 9$ реализация алгоритма полного перебора работает быстрее, чем реализация муравьиного алгоритма, но при дальнейшем увеличении размера матрицы реализация муравьиного алгоритма начинает выигрывать по времени. 

Заранее проведенная параметризация может помочь настроить муравьиный алгоритм так, что и он будет в большинстве случаев выдавать точный ответ, несмотря на то, что алгоритм предназначен для приближенного решения задачи коммивояжера.

Поставленная цель была достигнута: было разработано программное обеспечения, решающее задачу коммивояжера двумя способами: полным перебором и с помощью муравьиного алгоритма.