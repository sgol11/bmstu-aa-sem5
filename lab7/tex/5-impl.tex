\section{Технологический раздел \hfill}
\vspace{\baselineskip}

В данном разделе приводятся описание средств реализации и сами реализации алгоритмов, а также функциональные тесты.

\vspace{\baselineskip}
\subsection{Средства реализации}
\vspace{\baselineskip}

В качестве языка программирования для реализации лабораторной работы был выбран Python~\cite{PythonBook}. 
Данный язык предоставляет возможности работы с массивами и матрицами.

Для замера процессорного времени использовалась функция process\_time библиотеки time~\cite{process_time_text}.

\vspace{\baselineskip}
\subsection{Реализации алгоритмов}
\vspace{\baselineskip}

В листинге \ref{lst:full} приведена реализация алгоритма полного перебора для решения задачи коммивояжера.
В листингах \ref{lst:ant}---\ref{lst:ant4} приведена реализация муравьиного алгоритма для решения задачи коммивояжера.
\clearpage

\begin{lstlisting}[label=lst:full, caption=Реализация алгоритма полного перебора]
def full_search(matrix, size):
    all_ways = permutations(range(size))
    best_way = []
    min_len = float("inf")
    for way in all_ways:
        way = list(way) + [way[0]]
        cur_len = 0
        for i in range(size):
            cur_len += matrix[way[i]][way[i + 1]]
        if cur_len < min_len:
            best_way = way
            min_len = cur_len

    return best_way, min_len
\end{lstlisting}

\clearpage

\begin{lstlisting}[label=lst:ant, caption=Вычисление вероятностей перехода в следующий город]
def calc_prob(p_numerator, cur_city, size, root):
    res_probs = []

    for i in range(size):
        if i in root:
            res_probs.append(0)
        else:
            cur_prob = p_numerator[cur_city][i]
            res_probs.append(cur_prob)

    sum_probs = sum(res_probs)

    if sum_probs:
        for i in range(size):
            res_probs[i] /= sum_probs

    return res_probs
\end{lstlisting}

\begin{lstlisting}[label=lst:ant2, caption=Обновление матрицы обоняния tau]
def update_tau(tau):
    for i in range(size):
        for j in range(size):
            tau[i][j] *= (1 - p)
            tau[i][j] += delta_tau[i][j]
\end{lstlisting}

\clearpage

\begin{lstlisting}[label=lst:ant3, caption=Итерация муравьиного алгоритма]
def ant_iteration(matr, eta, tau, size, a, b, q, p, best, min_len):
    p_numerator = [[0] * size for _ in range(size)]
    for i in range(size):
        for j in range(size):
            p_numerator[i][j] = (eta[j][i] ** a) * (tau[j][i] ** b)
    delta_tau = [[0] * size for _ in range(size)]
    for i in range(size):
        way = [i]
        for j in range(size - 1):
            temp_probs = calc_prob(p_numerator, way[-1], size, way)
            prob = random()
            temp_sum, k = 0, 0
            while k < size:
                if temp_probs[k]:
                    temp_sum += temp_probs[k]
                    if temp_sum > prob:
                        break
                k += 1
            if k == size:
                k = size - 1
            way.append(k)
        way.append(way[0])
        cur_len = 0
        for x in range(len(way) - 1):
            cur_len += matr[way[x]][way[x + 1]]
        if cur_len < min_len:
            best = way
            min_len = cur_len
        for v in range(size):
            delta_tau[way[v]][way[v + 1]] += q / cur_len
    update_tau(tau)

    return tau, best_way, min_len
\end{lstlisting}

\begin{lstlisting}[label=lst:ant4, caption=Реализация муравьиного алгоритма]
def ant_algorithm(matrix, size, a, b, q, p, time):
    eta = [[0] * size for _ in range(size)]
    for i in range(size):
        for j in range(i):
            eta[i][j] = 1 / matrix[i][j]
            eta[j][i] = 1 / matrix[j][i]
    tau = [[0.2] * size for _ in range(size)]

    best_way = []
    min_len = float("inf")

    for t in range(time):
        tau, best_way, min_len = ant_iteration(matrix, eta, tau, size, a, b, q, p, best_way, min_len)

    return best_way, min_len
\end{lstlisting}

\vspace{\baselineskip}
\subsection{Тестирование}
\vspace{\baselineskip}

В таблице \ref{tabular:func_test} приведены функциональные тесты для реализаций алгоритмов, решающих задачу коммивояжера.

Тесты пройдены успешно каждой реализацией.

\clearpage

\begin{table}[h!]
	\begin{center}
	    \begin{threeparttable}
	    \captionsetup{justification=raggedright, singlelinecheck=off}
	    \caption{\label{tabular:func_test} Функциональное тестирование}
		\begin{tabular}{|c|c|c|}
			\hline
			Матрица смежности & Кратчайший путь & Длина пути \tabularnewline 
			\hline
			
			 &  & \tabularnewline[-1em]
			$\begin{pmatrix}
				0 & 1\\
				1 & 0
			\end{pmatrix}$ &
			  [0, 1, 0] &
			2
			\tabularnewline[2em]
			\hline
			
			 &  & \\[-1em]
			$\begin{pmatrix}
				0 & 1 & 2 & 3 \\
				1 & 0 & 1 & 2 \\
				2 & 1 & 0 & 1 \\
                3 & 2 & 1 & 0
			\end{pmatrix}$ &
			  [0, 1, 2, 3, 0] &
			6
			\tabularnewline[4em]
			\hline
			
			 &  & \\[-1em]
			$\begin{pmatrix}
				0 & 1 & INF & 2 \\
				1 & 0 & 2 & INF \\
				INF & 2 & 0 & 3 \\
                2 & INF & 3 & 0
			\end{pmatrix}$ &
			[0, 1, 2, 3, 0] &
			8
			\tabularnewline[4em]
			\hline
		\end{tabular}
		\end{threeparttable}
	\end{center}
\end{table}

\subsection*{Вывод}

Были описаны требования к программному обеспечению, средства реализации, приведены реализации алгоритмов и тесты, успешно пройденные программой.
