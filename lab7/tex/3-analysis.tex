\section{Аналитический раздел \hfill}
\vspace{\baselineskip}

В данном разделе представлено теоретическое описание задачи коммивояжера, а также алгоритма полного перебора и муравьиного алгоритма для ее решения.

\vspace{\baselineskip}
\subsection{Задача коммивояжера}
\vspace{\baselineskip}

В задаче коммивояжера рассматривается $n$ городов и матрица попарных расстояний между ними. 
Требуется найти такой порядок посещения городов, чтобы  суммарное пройденное расстояние было минимальным, каждый город посещался ровно один раз и коммивояжер вернулся в тот город, с которого начал свой маршрут.  
Другими словами, во взвешенном полном графе требуется найти гамильтонов цикл минимального веса~\cite{salesman}. 

\vspace{\baselineskip}
\subsection{Алгоритм полного перебора}
\vspace{\baselineskip}

Cуть алгоритма полного перебора для решения задачи коммивояжера заключается в переборе всех вариантов путей и нахождении кратчайшего.
Данный алгоритм имеет факториальную сложность $O(n!)$, что приводит к большим временным затратам даже при малых значениях числа вершин в графе.

\vspace{\baselineskip}
\subsection{Муравьиный алгоритм}
\vspace{\baselineskip}

Моделирование поведения муравьев связано с распределением феромона на тропе --- ребре графа в задаче коммивояжера. 
Более короткие пути сильнее обогащаются феромоном, вследствие чего являются более предпочтительными для всей колонии.
С помощью моделирования испарения феромона, т.е. отрицательной обратной связи, гарантируется, что найденное локально оптимальное решение не будет единственным: будут предприняты попытки поиска других путей.

Опишем правила поведения муравья при выборе пути применительно к решению задачи коммивояжера~\cite{shtovba}:

\begin{itemize}[label=---]
	\item муравьи обладают «памятью» в виде списка уже посещенных городов. Обозначим через $J_{i, k}$ список городов, которые необходимо посетить муравью k, находящемуся в городе i;
	\item муравьи обладают «зрением». Видимость есть эвристическое желание посетить город $j$, если муравей находится в городе $i$. Будем считать, что видимость обратно пропорциональна расстоянию между городами $i$ и $j$, обозначенному через $D_{ij}$. Так, $\eta_{ij} = \frac{1}{D_{ij}}$;
	\item муравьи обладают «обонянием». Они могут улавливать след феромона, подтверждающий желание посетить город $j$ из города $i$, на основании опыта других муравьев. Количество феромона на ребре $(i, j)$ в момент времени $t$ обозначим через $\tau_{ij}(t)$.
\end{itemize}

Вероятность перехода из города $i$ в город $j$ определяется по формуле~(\ref{eq:1}).
\begin{equation}
	\label{eq:1}
	P_{i,j}={\frac {(\tau_{i,j}^{\alpha})(\eta_{i,j}^{\beta })}{\sum (\tau_{i,j}^{\alpha})(\eta_{i,j}^{\beta})}},
\end{equation}
где  $\tau_{i,j}$ --- количество феромонов на ребре от $i$ до $j$;
$\eta_{i,j}$ --- эвристическое расстояние от $i$ до $j$;
$\alpha$ --- параметр влияния расстояния;
$\beta$ --- параметр влияния феромона.
При $\alpha = 0$ будет выбран ближайший город, что соответствует жадному алгоритму в классической теории оптимизации. Если $\beta=0$, работает лишь феромонное усиление, что влечет за собой быстрое вырождение маршрутов к одному субоптимальному решению.

После завершения движения всеми муравьями происходит обновление феромона.

Если $L_{k}$ --- длина пути $k$-ого муравья, $Q$ --- некоторая константа порядка длины путей, $N$ --- количество муравьев, $p \in [0, 1]$ --- коэффициент испарения феромона, то новое значения феромона на ребре $(i,j)$ рассчитывается по формуле~(\ref{eq:2}).
\begin{equation}\label{eq:2}
    \tau_{ij}(t+1) = (1-p)\tau_{ij}(t) + \Delta \tau_{ij},~~\Delta \tau_{ij} =
                     \displaystyle\sum_{k=1}^N \tau^k_{ij},
\end{equation}
где
\begin{equation}\label{eq:3}
    \Delta \tau^k_{ij} = \begin{cases}
        \frac{Q}{L_k}, & \quad \textrm{ребро посещено $k$-ым муравьем,} \\
        0, & \quad \textrm{иначе.}
    \end{cases}
\end{equation}

\vspace{\baselineskip}
\subsection*{Вывод}
\vspace{\baselineskip}

В данном разделе были рассмотрены идеи, необходимые для разработки и реализации двух алгоритмов решения задачи коммивояжера: алгоритма полного перебора и муравьиного алгоритма.
