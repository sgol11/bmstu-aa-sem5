\specsection{ВВЕДЕНИЕ \hspace{1.25cm}}
\vspace{\baselineskip}

В последние два десятилетия при оптимизации сложных систем исследователи все чаще применяют природные механизмы поиска наилучших решений. 
Один из таких механизмов --- это муравьиные алгоритмы, представляющие собой новый перспективный метод оптимизации, базирующийся на моделировании поведения колонии муравьев~\cite{shtovba}. 

Первый вариант муравьиного алгоритма был предназначен для приближенного решения задачи коммивояжера~\cite{ershov}. 

Целью данной работы является разработка программного обеспечения, решающего задачу коммивояжера двумя способами: полным перебором и с помощью муравьиного алгоритма.

В рамках выполнения работы необходимо решить следующие задачи: 
\begin{enumerate}[label=---]
	\item описать и реализовать алгоритм полного перебора для решения задачи коммивояжера;
	\item описать и реализовать муравьиный алгоритм для решения задачи коммивояжера;
	\item провести параметризацию муравьиного алгоритма на двух классах данных;
	\item провести сравнительный анализ времени выполнения и трудоемкостей реализаций.
\end{enumerate}