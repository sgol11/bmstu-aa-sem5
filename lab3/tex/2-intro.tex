\specsection{Введение}

Одной из важнейших процедур обработки информации является сортировка. 
Под сортировкой понимается упорядочивание элементов последовательности по какому-либо признаку \cite{book_shagbazyan}.

Алгоритмы сортировки имеют большое практическое применение. 
Их можно встретить там, где речь идет об обработке и хранении больших объемов информации. Некоторые задачи обработки данных решаются проще, если данные заранее упорядочить \cite{article_sortings}. 
С упорядоченными данными можно столкнуться в телефонных книгах, интернет-магазинах, библиотеках, словарях.

В настоящее время, в связи с постоянно растущими объемами данных и распространением BigData технологий, вопрос эффективности алгоритмов сортировки становится особенно актуальным \cite{big_data}.

Целью работы является получение навыков сравнения алгоритмов по трудоемкости и процессорному времени на примере трех алгоритмов сортировки.

В рамках выполнения работы необходимо решить следующие задачи:

\begin{itemize}
	\item изучить три алгоритма сортировки: пузырьком, вставками, блочной;
	\item описать и реализовать изученные алгоритмы;
	\item провести сравнительный анализ трудоемкости реализаций алгоритмов на основе теоретических расчетов;
	\item провести сравнительный анализ процессорного времени выполнения реализаций алгоритмов на основе экспериментальных данных.
\end{itemize}