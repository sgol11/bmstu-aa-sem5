\section{Технологический раздел}

В данном разделе будут приведены требования к программному обеспечению, средства реализации, листинги реализованных алгоритмов и тесты для программы.

\subsection{Требования к ПО}

Ниже представлен ряд требований к разрабатываемому программному обеспечению.

Требования к входным данным: массив состоит из целых чисел; длина массива может быть нулевой.

Требования к выходным данным: входной массив отсортирован по возрастанию каждым реализованным алгоритмом; на экран выведены графические результаты замеров времени сортировки каждым алгоритмом для трех случаев, которые подробно рассмотрены в пункте 4.3.

\subsection{Средства реализации}

В качестве языка программирования для реализации лабораторной работы был выбран Python \cite{PythonBook}. 
Данный язык предоставляет возможности работы с массивами.

Для замера процессорного времени использовалась функция process\_time библиотеки time \cite{process_time_text}.


\subsection{Реализации алгоритмов}
В листингах \ref{lst:bubble_sort}, \ref{lst:insert_sort}, \ref{lst:bucket_sort} представлены реализации алгоритмов сортировки (пузырьком, вставками и блочной сортировки). \newpage

\begin{lstlisting}[label=lst:bubble_sort,caption=Алгоритм сортировки пузырьком]
def bubble_sort(arr):
    n = len(arr)
    for i in range(n - 1):
        for j in range(n - i - 1):
            if arr[j] > arr[j + 1]:
                arr[j], arr[j + 1] = arr[j + 1], arr[j]
                
    return arr
\end{lstlisting}

\begin{lstlisting}[label=lst:insert_sort,caption= Алгоритм сортировки вставками]
def insertion_sort(arr):
    for i in range(1, len(arr)):
        x = arr[i]
        j = i - 1
        while j >= 0 and x < arr[j]:
            arr[j + 1] = arr[j]
            j -= 1
        arr[j + 1] = x

    return arr
\end{lstlisting}
\clearpage

\begin{lstlisting}[label=lst:bucket_sort,caption=Алгоритм блочной сортировки]
def bucket_sort(arr):
    max_a = max(arr)
    min_a = min(arr)
    n = len(arr)
    size = (max_a - min_a) / n

    buckets_list = []
    for x in range(n):
        buckets_list.append([])

    max_j = int(max_a / size)
    for i in range(n):
        j = n - max_j + int(arr[i] / size)
        if j != n:
            buckets_list[j].append(arr[i])
        else:
            buckets_list[n - 1].append(arr[i])

    arr = []
    for k in range(n):
        insertion_sort(buckets_list[k])
        arr = arr + buckets_list[k]

    return arr
\end{lstlisting}
\clearpage

\subsection{Тестирование}

В таблице \ref{tbl:functional_test} приведены тесты для функций, реализующих алгоритмы сортировки. 
Тесты пройдены успешно каждой реализацией алгоритма.


\begin{table}[h]
	\begin{center}
		\caption{\label{tbl:functional_test} Функциональные тесты}
		\begin{tabular}{|c|c|c|}
			\hline
			Входной массив & Ожидаемый результат & Результат \\ 
			\hline
			$[1,2,3,4,5]$ & $[1,2,3,4,5]$  & $[1,2,3,4,5]$\\
			$[1000,4,3,2,1]$  & $[1,2,3,4,1000]$ & $[1,2,3,4,1000]$\\
			$[2,3,1,5,4]$  & $[1,2,3,4,5]$  & $[1,2,3,4,5]$\\
			$[3,5,3,1,1]$  & $[1,1,3,3,5]$  & $[1,1,3,3,5]$\\
			$[1]$  & $[1]$  & $[1]$\\
			$[]$  & $[]$  & $[]$\\
			\hline
		\end{tabular}
	\end{center}
\end{table}


\subsection*{Вывод}

Были представлены листинги реализованных алгоритмов и тесты, успешно пройденные программой.