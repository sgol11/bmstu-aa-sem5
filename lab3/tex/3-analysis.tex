\section{Аналитический раздел}
В данном разделе будут рассмотрены основные идеи трех алгоритмов сортировки: пузырьком, вставками и блочной, а также будет произведена формализация задачи.

\subsection{Формализация задачи}

Пусть A -- массив, содержащий N чисел.
Массив считается отсортированным по возрастанию, если для каждого элемента массива выполняется следующее соотношение \ref{sort}:
\begin{equation}
	\label{sort}
		\forall i, j   \in [1, N] : i  \leq  j ; A_i \leq A_j 
\end{equation}

Массив считается отсортированным по убыванию, если для каждого элемента массива выполняется следующее соотношение \ref{back-sort}:
\begin{equation}
	\label{back-sort}
		\forall i, j   \in [1, N] : i  \leq  j ; A_i \geq A_j 
\end{equation}

В ходе выполнения работы требуется написать алгоритм, который сортирует массив из произвольных данных по возрастанию.

\subsection{Сортировка пузырьком}

Алгоритм  состоит  из  повторяющихся  проходов  по  сортируемому массиву. 
За каждый проход элементы последовательно сравниваются попарно, и, если порядок в паре неверный, выполняется обмен элементов. 
Проходы  по  массиву повторяются  $N-1$ раз.  
При каждом  проходе  алгоритма  по  внутреннему  циклу  очередной  наибольший  элемент  массива ставится на своё место в конце массива рядом с предыдущим «наибольшим элементом», а наименьший элемент перемещается на одну позицию к  началу  массива  («всплывает»  до  нужной  позиции,  как  пузырёк  в воде -- отсюда  и  название  алгоритма) \cite{book_shagbazyan}.

\subsection{Сортировка вставками}

На каждом  шаге  алгоритма  выбирается один  из  элементов входных данных и  помещается  на нужную  позицию  в  уже  отсортированной последовательности до тех  пор,  пока  набор  входных  данных  не будет исчерпан.  
Алгоритму  необходимо  для  каждого нового  элемента  выбрать нужное место для вставки в уже упорядоченный массив данных, так что элементы  входной  последовательности просматриваются  по одному,  и каждый новый поступивший элемент размещается  в  подходящее  место среди ранее  упорядоченных  элементов \cite{article_insert}.

\subsection{Блочная сортировка}

Блочная (корзинная или карманная) сортировка состоит в следующем. Пусть $l$ -- минимальный, а $r$ -- максимальный элемент массива. Разобьем элементы на блоки, в первом будут элементы от $l$ до $l + k$, во втором -- от $l + k$ до $l + 2k$ и т.д., где $k = (r - l)$ / количество блоков. 
Если количество блоков равно двум,
то данный алгоритм превращается в разновидность быстрой сортировки \cite{article_bucket}.


\subsection*{Вывод}

В данном разделе были рассмотрены идеи, лежащие в основе трех алгоритмов сортировки: пузырьком, вставками, блочной.