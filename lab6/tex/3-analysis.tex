\section{Аналитический раздел \hfill}
\vspace{\baselineskip}

В данном разделе приводятся определения словаря и лингвистической переменной, а также теоретическое описание алгоритма бинарного поиска. Кроме того, производится формализация объекта и его признаков.

\vspace{\baselineskip}
\subsection{Словарь}
\vspace{\baselineskip}

Словарь --- структура данных, которая позволяет хранить пары вида\linebreak <<ключ~---~значение>>: $(k, v)$. Ключ идентифицирует элемент словаря, значение является данными, которые соответствуют данному ключу. В словаре не может быть двух элементов с одинаковыми ключами, однако могут быть одинаковые значения у разных ключей~\cite{dict}.

\vspace{\baselineskip}
\subsection{Алгоритм бинарного поиска}
\vspace{\baselineskip}

Бинарный поиск производится в упорядоченном массиве.
При бинарном поиске искомый ключ сравнивается с ключом среднего элемента в массиве. 
Если они равны, то поиск успешен. 
В противном случае поиск осуществляется аналогично в левой или правой частях массива~\cite{binary}.

Пусть $n$ --- количество элементов в массиве. 
Тогда сложность алгоритма в худшем и среднем случае равна $O(\log{n})$, в лучшем --- $O(1)$.

У бинарного поиска есть недостаток: он требует упорядочивания данных по возрастанию. Сложность сортировки --- не менее $O(n\cdot\log{n})$. Поэтому бинарный поиск чаще применяется на массивах большого размера.

\vspace{\baselineskip}
\subsection{Лингвистическая переменная}
\vspace{\baselineskip}

Лингвистическая переменная характеризуется набором свойств $(X, T(X),$ $U, G, M)$, в котором
$X$ --- название переменной; 
$T(X)$ --- терм-множество переменной $X$, т.е. множество названий лингвистических значений переменной $X$, причем каждое из таких значений является нечеткой переменной $\Tilde{x}$ со значениями из универсального множества $U$ с базовой переменной $u$; 
$G$ --- синтаксическое правило, порождающее названия $\Tilde{x}$ значений переменной $X$; 
$M$ --- семантическое правило, которое ставит в соответствие каждой нечеткой переменной $\Tilde{x}$ ее смысл $M(\Tilde{x})$, т.е. нечеткое подмножество $M(\Tilde{x})$ универсального множества $U$.

Конкретное название $\Tilde{x}$, порожденное синтаксическим правилом $G$, называется термом. 
Терм, который состоит из одного слова или из нескольких слов, всегда фигурирующих вместе друг с другом, называется атомарным термом. 
Терм, который состоит из более чем одного атомарного терма, называется составным термом~\cite{ling1}.

\vspace{\baselineskip}
\subsection{Формализация объекта и его признаков} 
\vspace{\baselineskip}

Рассмотрим лингвистическую переменную с именем $X$ = <<Планеты Солнечной системы>>. 

Универсальное множество $U$ = [<<Меркурий>>, <<Венера>>, <<Марс>>, <<Юпитер>>, <<Сатурн>>, <<Уран>>, <<Нептун>>].

Терм-множество $T$ = [<<Близко>>, <<Средне>>, <<Далеко>>]. 
Данные термы показывают дальность расположения конкретной планеты относительно Земли.

Синтаксическое правило $G$ порождает новые термы с использованием квантификаторов <<не>> и <<очень>>.
\clearpage
$M$ --- процедура, ставящая каждому новому терму в соответствие нечеткое множество из $X$ по правилам: если терм $t$ имел функцию принадлежности $\mu_t(u)$, то новые термы будут иметь функции принадлежности, заданные в таблице \ref{tabular:terms}.

\begin{table}[h!]
	\begin{center}
	    \begin{threeparttable}
	    \captionsetup{justification=raggedright, singlelinecheck=off}
	    \caption{\label{tabular:terms} Функции принадлежности составных термов}
		\begin{tabular}{|>{\centering}p{0.4\linewidth}|p{0.5\linewidth}<{\centering}|}
			\hline
			Квантификатор & Функция принадлежности \tabularnewline 
                \hline
                не $t$ & $1 - \mu_t(u)$ \\
                \hline
                очень $t$ & $(\mu_t(u))^2$ \\
                \hline
		\end{tabular}
		\end{threeparttable}
	\end{center}
\end{table}

\subsection{Требования к программному обеспечению}
\vspace{\baselineskip}

Программа должна предоставлять следующие возможности:
\begin{itemize}[label=---]
    \item ввод запроса;
    \item вывод на экран найденных в словаре объектов, удовлетворяющих ограничению, заданному в вопросе.
\end{itemize}

\vspace{\baselineskip}
\subsection*{Вывод}
\vspace{\baselineskip}

Были рассмотрены идеи, необходимые для разработки и реализации алгоритма бинарного поиска, а также определения терминов, используемых в лабораторной работе.
