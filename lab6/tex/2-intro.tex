\specsection{ВВЕДЕНИЕ \hspace{1.25cm}}
\vspace{\baselineskip}

Способность человека оценивать информацию играет существенную\linebreak роль в определении сложных явлений~\cite{ling2}. 
В лабораторной работе рассматривается понятие лингвистической переменной, которая отличается от числовой тем, что ее значениями являются не числа, а слова или предложения в естественном или формальном языке. 
Поскольку слова в общем менее точны, чем числа, понятие лингвистической переменной дает возможность приближенно описывать явления, которые настолько сложны, что не поддаются описанию в общепринятых количественных терминах~\cite{ling1}.

Словарь --- это структура данных, позволяющая идентифицировать ее элементы не по числовому индексу, а по произвольному.
Одной из основных операций над рассматриваемой структурой данных является поиск.

Целью данной работы является разработка метода поиска по словарю при ограничении на значение признака, заданном при помощи лингвистической переменной.

В рамках выполнения работы необходимо решить следующие задачи: 
\begin{enumerate}[label=---]
        \item формализовать объект и его признаки;
        \item провести анкетирование респондентов; 
        \item описать алгоритм поиска в словаре объектов, удовлетворяющих ограничению, заданном в вопросе на ограниченном естественном языке;
	\item реализовать описанный алгоритм поиска в словаре;
        \item привести примеры запросов пользователя и сформированной реализацией алгоритма поиска выборки объектов из словаря, используя составленные респондентами вопросы;
        \item дать заключение о применимости предложенного алгоритма.
\end{enumerate}