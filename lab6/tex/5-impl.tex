\section{Технологический раздел \hfill}
\vspace{\baselineskip}

В данном разделе приводятся описание средств реализации и сами реализации алгоритмов, а также функциональные тесты.

\vspace{\baselineskip}
\subsection{Средства реализации}
\vspace{\baselineskip}

В качестве языка программирования для реализации лабораторной работы был выбран Python~\cite{PythonBook}. 
Данный язык предоставляет возможности работы с массивами и словарями.
Кроме того, в языке есть морфологический анализатор pymorphy2, позволяющий получать начальные формы слов.

\vspace{\baselineskip}
\subsection{Представление данных}
\vspace{\baselineskip}

В реализации алгоритма получения объектов, удовлетворяющих заданному ограничению, используются переменные distance и planets\_dist.

distance представлет собой массив из 3 массивов, каждый из которых содержит синонимы конкретного терма (<<близко>>, <<средне>> или <<далеко>>).

planets\_dist является словарем, в котором ключами выступают планеты, а значенями --- массивы из 3 элементов, содержащие значения функций принадлежности для каждого терма.

Сам словарь, в котором осуществляется поиск, является аналогией обычного бумажного словаря, в котором в качестве ключа выступает слово, а в качестве значения --- страница. Слова и страницы генерируются случайным образом, но среди слов обязательно есть названия всех планет Солнечной системы. 

\vspace{\baselineskip}
\subsection{Реализации алгоритмов}
\vspace{\baselineskip}

В листинге \ref{lst:bin} приведена реализация алгоритма бинарного поиска в словаре.
В листинге \ref{lst:not_very} --- реализация алгоритмов получения объектов по значению функции принадлежности, а также вычисления значений этой функции для квантификаторов <<не>> и <<очень>>.
В листинге \ref{lst:get} --- реализация алгоритма получения объектов по заданному на естественном языке ограничению.
\clearpage

\captionsetup{justification=raggedright, singlelinecheck=off}
\begin{lstlisting}[label=lst:bin, caption=Реализация алгоритма бинарного поиска]
def sort_dict(my_dict):
    keys = list(my_dict.keys())
    keys.sort()
    tmp_dict = dict()
    for key in keys:
        tmp_dict[key] = my_dict[key]

    return tmp_dict

def binary_search(sort_dict, find_key, output=True):
    result = -1
    keys = list(sort_dict.keys())
    left, middle, right = 0, len(keys) // 2, len(keys) - 1
    while left <= right:
        key = keys[middle]
        if key == find_key:
            if output:
                print(list(sort_dict.items())[middle])
            result = 0
            break
        elif key < find_key:
            left = middle + 1
        else:
            right = middle - 1
        middle = (left + right) // 2

    return result

def process_binary_search(global_dict, find_key):
    sorted_dict = sort_dict(global_dict)
    result = binary_search(sorted_dict, find_key)
    if result == -1:
        print("Key not found")
\end{lstlisting}

\clearpage

\begin{lstlisting}[label=lst:not_very, caption=Реализация алгоритмов вычисления значений функций принадлежности]
def get_planets_by_dist(pl_dist, t):
    res = []
    for planet in pl_dist.keys():
        if pl_dist[planet][t] > pl_dist[planet][t - 1] and \
           pl_dist[planet][t] > pl_dist[planet][t - 2]:
            res.append(planet)

    return res
    
def inv(a):
    return 1 - a

def special_not(pl_dist, t):
    res = []
    for planet in pl_dist.keys():
        if inv(pl_dist[planet][t]) > pl_dist[planet][t]:
            res.append(planet)
        pl_dist[planet][t] = inv(pl_dist[planet][t])

    return res

def special_very(pl_dist, t):
    res = []
    for planet in pl_dist.keys():
        if (pl_dist[planet][t])**2 > inv((pl_dist[planet][t])**2):
            res.append(planet)
        pl_dist[planet][t] = (pl_dist[planet][t]) ** 2

    return res
\end{lstlisting}

\clearpage

\begin{lstlisting}[label=lst:get, caption=Реализация алгоритма получения удовлетворяющих заданному ограничению объектов]
def get_planets(question_split, distance, planets_dist):
    res = -1
    special_words = []

    for word in question_split:
        n_form = morph.parse(word)[0].normal_form
        for i in range(3):
            if n_form in distance[i]:
                res = i
        if n_form == not_word:
            special_words.append('not')
        if n_form == very_word:
            special_words.append('very')

    if res == -1:
        print("\nThe question is not recognized :(")
    else:
        planets = []
        special_words.reverse()

        if not special_words:
            planets.extend(get_planets_by_dist(planets_dist, res))
        else:
            tmp = []
            for i in special_words:
                tmp = []
                if i == 'not':
                    tmp.extend(special_not(planets_dist, res))
                elif i == 'very':
                    tmp.extend(special_very(planets_dist, res))
            planets.extend(tmp)
\end{lstlisting}
\captionsetup{justification=centering}

\vspace{\baselineskip}
\subsection{Тестирование}
\vspace{\baselineskip}

В таблице \ref{tabular:func_test} приведены функциональные тесты. 
Тесты были успешно пройдены программой.

\begin{table}[h!]
	\begin{center}
	    \begin{threeparttable}
	    \captionsetup{justification=raggedright, singlelinecheck=off}
	    \caption{\label{tabular:func_test} Функциональное тестирование}
		\begin{tabular}{|c|c|}
			\hline
			Запрос & Ожидаемый ответ \\
                \hline
                Какие планеты находятся рядом с Землей? & Венера, Марс \\
                \hline
                Планеты на среднем расстоянии от Земли & Меркурий, Юпитер \\
                \hline
                планеты очень далеко от земли & Уран, Нептун \\
                \hline
                \multirow{2}*{планеты не далеко от земли} & Меркурий, Венера, Марс, \\
                & Юпитер \\
                \hline
                планеты очень не далеко от земли & Венера, Марс \\
                \hline
                \multirow{2}*{планеты не очень далеко от земли} & Меркурий, Венера, Марс, \\
                & Юпитер, Сатурн \\
                \hline
                самая дальняя звезда & Вопрос не распознан :( \\
			\hline
                борщ рецепт & Вопрос не распознан :( \\
			\hline
		\end{tabular}
		\end{threeparttable}
	\end{center}
\end{table}

\subsection*{Вывод}
\vspace{\baselineskip}

Были описаны средства реализации, приведены реализации алгоритмов и тесты, успешно пройденные программой.
