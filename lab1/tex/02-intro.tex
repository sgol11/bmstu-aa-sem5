\specsection{ВВЕДЕНИЕ \hspace{1.25cm}}
\vspace{\baselineskip}

Расстояние Левенштейна -- мера, которая определяет, насколько две строки различаются между собой. Фактически данная величина определяет, сколько односимвольных изменений (вставки, удаления, замены) требуется для преобразования одной последовательности символов к другой.

Модификацией расстояния Левенштейна является расстояние Дамерау-Левенштейна, в котором к операциям вставки, удаления и замены добавляется операция транспозиции (перестановки) символов. Данную меру чаще всего используют, когда текст набирается с клавиатуры и возрастает вероятность ошибки перестановки двух соседних симолов.

Расстояния Левенштейна и Дамерау-Левенштейна нашли широкое применение в следующих сферах:
\begin{itemize}
    \item компьютерная лингвистика (автоматическое исправление ошибок в тексте, обнаружение возможных ошибок в поисковых запросах, расчет изменений в различных версиях текста);
    \item биоинформатика (сравнение последовательностей генов).
\end{itemize}

Целью работы является разработка, реализация и исследование алгоритмов нахождения расстояний Левенштейна и Дамерау-Левенштейна.

В рамках выполнения работы необходимо решить следующие задачи:

\begin{itemize}
	\item изучить расстояния Левенштейна и Дамерау-Левенштейна;
	\item разработать и реализовать алгоритмы нахождения изученных расстояний;
	\item провести сравнительный анализ процессорного времени выполнения реализаций данных алгоритмов;
	\item провести сравнительный анализ максимальной затрачиваемой алгоритмами памяти.
\end{itemize}