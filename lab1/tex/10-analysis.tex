\section{Аналитический раздел \hfill}
\vspace{\baselineskip}

В данном разделе будут представлены описания редакционных расстояний Левенштейна и Дамерау-Левенштейна, а также варианты реализации поиска этих расстояний.

\subsection{Расстояние Левенштейна}

Расстояние Левенштейна \cite{Levenshtein} -- это минимальное количество редакторских операций, необходимых для преобразования одной строки в другую.

Используются следующие редакторские операции:
\begin{itemize}
    \item I (англ. insert) — вставка;
	\item D (англ. delete) — удаление;
	\item R (англ. replace) — замена.
\end{itemize}

Операциям I, D и R назначают цену (штраф) 1. Также существует обозначение M (англ. match) -- совпадение символов. Штраф M составляет 0, т.к. в случае совпадения символов никаких действий не производится.

Задача нахождения расстояния Левенштейна сводится к поиску последовательности действий, минимизирующих суммарный штраф.


Пусть $S_{1}$ и $S_{2}$ — две строки (длиной M и N соответственно) над некоторым алфавитом. Тогда расстояние Левенштейна можно подсчитать по следующей рекуррентной формуле:

\begin{equation}
	\label{eq:D}
	D(i, j) = \begin{cases}
		
		0 &\text{i = 0, j = 0}\\
		i &\text{j = 0, i > 0}\\
		j &\text{i = 0, j > 0}\\
		\min \lbrace \\
		\qquad D(i, j-1) + 1,\\
		\qquad D(i-1, j) + 1, &\text{i > 0, j > 0}\\
		\qquad D(i-1, j-1) + m(S_1[i], S_2[j]).\\
		\rbrace
	\end{cases}
\end{equation}

Функция m определена как:
\begin{equation}
	\label{eq:m}
	m(a, b) = \begin{cases}
		0, &\text{a = b},\\
		1, &\text{иначе}.
	\end{cases}
\end{equation}

\subsubsection{Нерекурсивный алгоритм для определения расстояния Левенштейна (с использованием матрицы расстояний)}
Введем матрицу размером $(length(S1)+ 1)$ x $((length(S2) + 1)$, где $length(S)$ — длина строки S. Значение в ячейке $[i, j]$ равно значению\newline
$D(S1[1...i], S2[1...j])$. Первая строка и первый столбец тривиальны. 

Всю таблицу (за исключением первого столбца и первой строки) заполняем в соответствии с формулой \ref{eq:mat}.
\begin{equation}
	\label{eq:mat}
	A[i][j] = min \begin{cases}
		A[i-1][j] + 1,\\
		 A[i][j-1] + 1,\\
		 A[i-1][j-1] + m(S1[i], S2[j]).\\
	 \end{cases}
 \end{equation}

Функция m определена как:
\begin{equation}
\label{eq:m2}
m(S1[i], S2[j]) = \begin{cases}
0, &\text{если $S1[i] = S2[j]$,}\\
1, &\text{иначе.}
\end{cases}
\end{equation}
 
В результате расстоянием Левенштейна будет ячейка матрицы с индексами $i = length(S1$) и $j = length(S2)$.

\subsection{Расстояние Дамерау-Левенштейна}
Расстояние Дамерау-Левенштейна является модификацией алгоритма Левенштейна и чаще всего используется при наборе текста с клавиатуры, т.к. в этом случае пользователь может допустить опечатку, переставив местами два соседних символа. 

К существующим редакторским операциям добавляется еще одна - X (англ. Exchange), штраф за которую также составляет 1. Тогда функция $D$ вычисляется по формуле \ref{eq:DL}.

\begin{equation}
	\label{eq:DL}
	D(i, j) = \begin{cases}
		0 &\text{i = 0, j = 0}\\
		i &\text{j = 0, i > 0}\\
		j &\text{i = 0, j > 0}\\
		\min \lbrace \\
			\qquad D(i, j-1) + 1,\\
			\qquad D(i-1, j) + 1, &\text{i > 0, j > 0}\\
			\qquad D(i-1, j-1) + m(a[i], b[j]),\\
			\qquad \left[ \begin{array}{cc}D(i-2, j-2) + 1, &\text{если }i,j > 1;\\
			\qquad &\text{}a[i] = b[j-1]; \\
			\qquad &\text{}b[j] = a[i-1]\\
			\qquad \infty, & \text{иначе}\end{array}\right.\\
		\rbrace
	\end{cases}
\end{equation}

\subsubsection{Рекурсивный алгоритм для определения расстояния Дамерау"=Левенштейна}
Рекурсивный вариант напрямую реализует формулу \ref{eq:DL}, вызывая функцию $D$ для величин, равных длине каждой из строк. 
Сразу можно отметить недостаток данной реализации: одно и то же значение будет подсчитано не\-сколько раз, поэтому будет возникать проблема повторных вычислений.

\subsubsection{Рекурсивный алгоритм для определения расстояния Дамерау"=Левенштейна с кешированием}
В качестве оптимизации предыдущего алгоритма можно подсчитанные значения сохранять в матрицу, в которой строкам i и j будет соответствовать значение функции $D(i, j)$. Каждый раз при подсчёте $D$ программа будет проверять, было ли уже подсчитано значение для заданных аргументов, и если да, то будет использовать уже готовый вариант.

\subsubsection{Нерекурсивный алгоритм для определения расстояния Дамерау"=Левенштейна}
Нерекурсивный алгоритм для определения расстояния Дамерау"=Левенштейна с использованием матрицы аналогичен нерекурсивному алгоритму для определения расстояния Левенштейна. Добавляется лишь последнее ус\-ловие из формулы \ref{eq:DL}.

\subsection*{Вывод}
Формулы для определения расстояния Левенштейна и Дамерау"=Левенштейна между строками задаются рекуррентно, а следовательно, алгоритмы могут быть реализованы рекурсивно или итерационно.

В данном разделе были рассмотрены идеи итеративной реализации алгоритма для нахождения расстояния Левенштейна, а также идеи рекурсивной, рекурсивной с кешированием и итеративной реализации алгоритма для нахождения расстояния Дамерау-Левенштейна. 
