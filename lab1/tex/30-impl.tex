\section{Технологический раздел \hfill}
\vspace{\baselineskip}

В данном разделе будут приведены требования к программному обеспечению, средства реализации, листинги реализованных алгоритмов и тесты для программы.

\subsection{Требования к программному обеспечению}

Требования к входным данным:
\begin{itemize}
	\item на вход подаются две строки;
	\item буквы верхнего и нижнего регистров считаются различными;
	\item строки могут быть пустыми.
\end{itemize}

Требования к выходным данным:
\begin{itemize}
	\item искомое расстояние должно быть вычислено каждым из рассматриваемых алгоритмов;
	\item в режиме отладки должна быть выведена матрица расстояний.
\end{itemize}

\subsection{Средства реализации}

В качестве языка программирования для реализации лабораторной работы был выбран язык Python \cite{PythonBook}. Данный язык предоставляет возможности работы со строками и матрицами.

Для замера процессорного времени использовалась функция библиотеки time process\_time \cite{process_time_text}.


\subsection{Реализации алгоритмов}
В листингах \ref{lst:l-matr} -- \ref{lst:dl-rec-opt} представлены реализации рассматриваемых алгоритмов поиска редакционного расстояния. 
\clearpage

\begin{lstlisting}[label=lst:l-matr,caption=Функция нерекурсивного поиска расстояния Левенштейна]
def lowenstein_dist_matrix(str1, str2):
    str1 = ' ' + str1
    str2 = ' ' + str2
    n = len(str1)
    m = len(str2)
    matrix = [[0] * m for _ in range(n)]

    for i in range(1, n):
        matrix[i][0] = i  
    for j in range(1, m):
        matrix[0][j] = j

    for i in range(1, n):
        for j in range(1, m):
            insert = matrix[i][j-1] + 1
            delete = matrix[i-1][j] + 1
            replace = matrix[i-1][j-1] + int(str1[i] != str2[j])

            matrix[i][j] = min(insert,delete,replace)

    return matrix[n - 1][m - 1]
\end{lstlisting}
\clearpage

\begin{lstlisting}[label=lst:dl-matr,caption=Функция нерекурсивного поиска расстояния Дамерау- Левенштейна]
def damerau_lowenstein_dist_matrix(str1, str2):
    str1 = ' ' + str1
    str2 = ' ' + str2
    n = len(str1)
    m = len(str2)
    matrix = [[0] * m for _ in range(n)]

    for i in range(1, n):
        matrix[i][0] = i
    for j in range(1, m):
        matrix[0][j] = j 

    for i in range(1, n):
        for j in range(1, m):
            insert = matrix[i][j-1] + 1
            delete = matrix[i-1][j] + 1
            replace = matrix[i-1][j-1] + int(str1[i] != str2[j])
            if (i > 1 and j > 1) and (str1[i] == str2[j - 1] and str1[i-1] == str2[j]):
                xchange = matrix[i-2][j - 2] + 1
                matrix[i][j] = min(insert,delete,replace,xchange)
            else:
                matrix[i][j] = min(insert,delete,replace)

    return matrix[n - 1][m - 1]
\end{lstlisting}
\clearpage

\begin{lstlisting}[label=lst:dl-rec,caption=Функция рекурсивного алгоритма поиска расстояния Дамерау-Левенштейна без кеширования]
def dam_lowenstein_dist_rec(str1, str2):
    n = len(str1)
    m = len(str2)

    if n == 0 or m == 0:
        return n + m

    insert = dam_lowenstein_dist_rec(str1, str2[:-1]) + 1
    delete = dam_lowenstein_dist_rec(str1[:-1], str2) + 1
    replace = dam_lowenstein_dist_rec(str1[:-1], str2[:-1]) + \ 
                                      int(str1[-1] != str2[-1])

    if (n > 1 and m > 1) and \
       (str1[-1] == str2[-2] and str1[-2] == str2[-1]):
        xchange = dam_lowenstein_dist_rec(str1[:-2], str2[:-2])+1
        return min(insert,delete,replace,xchange)
    else:
        return min(insert,delete,replace)
\end{lstlisting}
\clearpage

\begin{lstlisting}[label=lst:dl-rec-opt,caption=Функция рекурсивного алгоритма поиска расстояния Дамерау-Левенштейна c кешированием]
def dam_lowenstein_dist_rec_optimized(str1, str2):
    def rec(str1, str2, a):
        len1 = len(str1)
        len2 = len(str2)
        if len1 == 0 or len2 == 0:
            a[len1][len2] = len1 + len2
        else:
            if a[len1][len2 - 1] == -1:
                rec(str1, str2[:-1], a)
            if a[len1 - 1][len2] == -1:
                rec(str1[:-1], str2, a)
            if a[len1 - 1][len2 - 1] == -1:
                rec(str1[:-1], str2[:-1], a)

            if (len1>1 and len2>1) and (str1[-1] == str2[-2] 
                                   and str1[-2] == str2[-1]):
                if a[len1-2][len2-2] == -1:
                    rec(str1[:-2], str2[:-2], a)
                a[len1][len2] = min(a[len1][len2-1]+1,
                a[len1-1][len2]+1, a[len1-2][len2-2]+1,
                a[len1-1][len2-1] + int(str1[-1] != str2[-1]))
            else:
                a[len1][len2] = min(a[len1][len2-1] + 1,
                a[len1-1][len2] + 1, 
                a[len1-1][len2 - 1] + int(str1[-1] != str2[-1]))
        return
        
    n = len(str1) + 1
    m = len(str2) + 1
    matrix = [[-1] * m for _ in range(n)]
    rec(str1, str2, matrix)
    return matrix[n - 1][m - 1]
\end{lstlisting}


\subsection{Тестирование}

В таблице \ref{test} приведены функциональные тесты для алгоритмов вычисления расстояний Левенштейна и Дамерау—Левенштейна.

В таблице приняты обозначения: РЛ - алгоритм поиска расстояния Левенштейна, РДЛ - алгоритм поиска расстояния Дамерау-Левенштейна. 

\begin{table}[h]
    \captionsetup{justification=ragged_right}
	\begin{center}
	    \begin{threeparttable}
	    \captionsetup{justification=raggedright, singlelinecheck=off}
		\caption{\label{test} Функциональные тесты}
		\begin{tabular}{|>{\centering}p{0.2\linewidth}|>{\centering}p{0.2\linewidth}|p{0.2\linewidth}<{\centering}|p{0.2\linewidth}<{\centering}|}
			\hline
			\multirow{2}*{Строка 1} & \multirow{2}*{Строка 2} & \multicolumn{2}{c|}{Ожидаемый результат} \\ \cline{3-4}
			& & РЛ & РДЛ \\ 
			\hline
			 &  & 0 & 0 \\
			\hline
			кот & скат & 2 & 2 \\
			\hline
			осень & очеьн & 3 & 2 \\
			\hline
			сон & сноп & 2 & 2 \\
			\hline
			мир & мира & 1 & 1 \\
			\hline
			мира & мир & 1 & 1 \\
			\hline
			рим & ром & 1 & 1 \\
			\hline
			дождь & длджь & 3 & 2 \\
			\hline
		\end{tabular}
		\end{threeparttable}
	\end{center}
\end{table}
Все тесты пройдены успешно.

\subsection*{Вывод}

Был произведен выбор средств реализации, реализованы и протестированы алгоритмы поиска расстояний: Левенштейна -- итерационный, Дамерау-Левенштейна -- итерационный и рекурсивный (с кешем и без).