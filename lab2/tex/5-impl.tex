\section{Технологический раздел \hfill}
\vspace{\baselineskip}

\subsection{Требования к программному обеспечению}

На вход программе подаются две матрицы, а на выходе должно быть получено искомое произведение матриц, вычисленное с помощью каждого реализованного алгоритма: стандартного, Винограда и Винограда с оптимизациями.

\subsection{Средства реализации}

В качестве языка программирования для реализации лабораторной работы был выбран язык Python \cite{PythonBook}. Данный язык предоставляет возможности работы с массивами и матрицами.

Для замера процессорного времени использовалась функция process\_time библиотеки time \cite{process_time_text}.


\subsection{Реализации алгоритмов}
В листингах \ref{lst:standard}, \ref{lst:vinograd}, \ref{lst:vinograd_opt} представлены реализации алгоритмов умножения матриц: стандартного, Винограда, Винограда с оптимизациями.
\newpage

\begin{lstlisting}[label=lst:standard, caption=Стандартный алгоритм умножения матриц]
def standard_matrix_mult(matr1, matr2):
    m, n, q = get_correct_matrices_sizes(matr1, matr2)
    result_matr = [[0] * q for _ in range(m)]
    
    for i in range(m):
        for j in range(q):
            for k in range(n):
                result_matr[i][j] = result_matr[i][j] + \ 
                                    matr1[i][k] * matr2[k][j]

    return result_matr
\end{lstlisting}
\newpage

\begin{lstlisting}[label=lst:vinograd,caption=Алгоритм Винограда умножения матриц]
def vinograd_matrix_mult(matr1, matr2):
    m, n, q = get_correct_matrices_sizes(matr1, matr2)
    result_matr = [[0] * q for _ in range(m)]

    row = [0] * m
    for i in range(m):
        for j in range(n // 2):
            row[i] = row[i] + matr1[i][j*2] * matr1[i][j*2+1]

    col = [0] * q
    for i in range(q):
        for j in range(n // 2):
            col[i] = col[i] + matr2[j*2][i] * matr2[j*2+1][i]

    for i in range(m):
        for j in range(q):
            result_matr[i][j] = -row[i] - col[j]
            for k in range(n // 2):
                result_matr[i][j] = result_matr[i][j] + \
                            (matr1[i][k*2] + matr2[2*k+1][j]) * \
                            (matr1[i][2*k+1] + matr2[2*k][j])

    if n % 2 == 1:
        for i in range(m):
            for j in range(q):
                result_matr[i][j] = result_matr[i][j] + \ 
                                    matr1[i][n-1] * matr2[n-1][j]

    return result_matr
\end{lstlisting}
\clearpage

\begin{lstlisting}[label=lst:vinograd_opt,caption=Оптимизированный алгоритм Винограда умножения матриц]
def vinograd_matrix_mult_optimized(matr1, matr2):
    m, n, q = get_correct_matrices_sizes(matr1, matr2)
    result_matr = [[0] * q for _ in range(m)]

    row = [0] * m
    for i in range(m):
        for j in range(n // 2):
            row[i] += matr1[i][j << 1] * matr1[i][j << 1 + 1]

    col = [0] * q
    for i in range(q):
        for j in range(n // 2):
            col[i] += matr2[j << 1][i] * matr2[j << 1 + 1][i]

    for i in range(m):
        for j in range(q):
            tmp = -row[i] - col[j]
            for k in range(n // 2):
                tmp += (matr1[i][k << 1] + matr2[k << 1 + 1][j]) *\
                       (matr1[i][k << 1 + 1] + matr2[k << 1][j])
            result_matr[i][j] = tmp

    if n % 2 == 1:
        for i in range(m):
            for j in range(q):
                result_matr[i][j] += matr1[i][n-1] * matr2[n-1][j]

    return result_matr
\end{lstlisting}
\clearpage

\subsection{Тестирование}

В таблице \ref{tabular:func_test} приведены функциональные тесты для функций, реализующих алгоритмы умножения матриц. 

Все тесты пройдены успешно каждой реализацией алгоритма.

\begin{table}[h!]
	\begin{center}
	    \begin{threeparttable}
	    \captionsetup{justification=raggedright, singlelinecheck=off}
	    \caption{\label{tabular:func_test} Тестирование функций}
		\begin{tabular}{|c|c|c|}
			\hline
			Матрица A & Матрица B & Ожидаемый результат C \tabularnewline 
			\hline
			
			 &  & \tabularnewline[-1em]
			$\begin{pmatrix}
				2
			\end{pmatrix}$ &
			$\begin{pmatrix}
				2
			\end{pmatrix}$ &
			$\begin{pmatrix}
				4
			\end{pmatrix}$
			\tabularnewline[1em]
			\hline
			
			 &  & \\[-1em]
			$\begin{pmatrix}
				1 & 1\\
				1 & -1\\
				2 & 2
			\end{pmatrix}$ &
			$\begin{pmatrix}
				0 & -1 & 1 & 2\\
				0 & 1 & 1 & 3
			\end{pmatrix}$ &
			$\begin{pmatrix}
				0 & 0 & 2 & 5\\
				0 & -2 & 0 & -1\\
				0 & 0 & 4 & 10
			\end{pmatrix}$
			\tabularnewline[3em]
			\hline
			
			 &  & \\[-1em]
			$\begin{pmatrix}
				1 & 1 & 1\\
				1 & 1 & 1\\
				1 & 1 & 1
			\end{pmatrix}$ &
			$\begin{pmatrix}
				1 & 2 & 3\\
				4 & 5 & 6\\
				7 & 8 & 9
			\end{pmatrix}$ &
			$\begin{pmatrix}
				12 & 15 & 18\\
				12 & 15 & 18\\
				12 & 15 & 18
			\end{pmatrix}$
			\tabularnewline[3em]
			\hline
			
			 &  & \\[-1em]
			$\begin{pmatrix}
				1 & 1\\
				2 & 2
			\end{pmatrix}$ &
			$\begin{pmatrix}
				1
			\end{pmatrix}$ &
			Не могут быть перемножены 
			\tabularnewline[2em]
			\hline
		\end{tabular}
		\end{threeparttable}
	\end{center}
\end{table}

\subsection*{Вывод}

Были представлены листинги реализованных алгоритмов и тесты, успешно пройденные программой.
