\specsection{ВВЕДЕНИЕ \hspace{1.25cm}}
\vspace{\baselineskip}

Матрица -- математический объект, записываемый в виде прямоугольной таблицы элементов кольца или поля (например, целых или комплексных чисел), которая представляет собой совокупность строк и столбцов, на пересечении которых находятся её элементы. Количество строк и столбцов матрицы задают размер матрицы \cite{article_matrix}.

Матричные вычисления -- основа многих алгоритмов. Матрицы получили широкое распространение в компьютерной графике, моделировании физических экспериментов, цифровой обработке изображений, моделировании графов. Зачастую над матрицами приходится производить некоторые операции, одной из которых является умножение.

Умножение матриц -- затратная операция, поэтому возникает необходимость в подборе алгоритмов, позволяющих оптимизировать это действие.

Целью работы является изучение способов оптимизации алгоритмов на примере алгоритмов умножения матриц.

В рамках выполнения работы необходимо решить следующие задачи:

\begin{itemize}[label=---]
	\item изучить алгоритмы умножения матриц;
	\item описать алгоритмы умножения матриц --- стандартный и Винограда;
	\item разработать алгоритм Винограда с оптимизациями;
	\item реализовать три алгоритма умножения матриц --- стандартный, Винограда и Винограда с оптимизациями;
	\item оценить трудоемкость алгоритмов;
	\item провести сравнительный анализ процессорного времени выполнения реализаций алгоритмов.
\end{itemize}