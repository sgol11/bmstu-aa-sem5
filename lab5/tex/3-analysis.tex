\section{Аналитический раздел \hfill}
\vspace{\baselineskip}

\numberwithin{equation}{subsection}

\subsection{Конвейерная обработка данных}

Конвейеризация (или конвейерная обработка) в общем случае основана на разделении подлежащей исполнению функции на более мелкие части, называемые ступенями, и выделении для каждой из них отдельного блока аппаратуры. Так обработку любой машинной команды можно разделить на несколько этапов (несколько ступеней), организовав передачу данных от одного этапа к следующему. При этом конвейерную обработку можно использовать для совмещения этапов выполнения разных команд. Производительность при этом возрастает благодаря тому, что одновременно на различных ступенях конвейера выполняются несколько команд \cite{conveyor}. 

\subsection{Раскраска графа по количеству смежных вершин}

Раскраска вершин графа $G(V, E)$ в $k$ различных цветов является отображением множества его вершин $V$ во множество ${1,2,\dots,k}, k\in N$, в котором каждый элемент интерпретируется как цвет вершины.

В данной работе рассматривается раскраска графа в зависимости от количества соседей каждой вершины: чем больше смежных вершин у конкретной вершины, тем в более <<горячий>> цвет она окрашивается.

\subsection*{Вывод}

В данном разделе были рассмотрены идеи, необходимые для разработки и реализации линейного и параллельного конвейерного вариантов раскраски графа.
