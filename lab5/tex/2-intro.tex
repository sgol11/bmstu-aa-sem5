\specsection{ВВЕДЕНИЕ \hspace{1.25cm}}
\vspace{\baselineskip}

Разработчики архитектуры компьютеров издавна прибегали к методам проектирования, известным под общим названием "совмещение операций", при котором аппаратура компьютера в любой момент времени выполняет одновременно более одной базовой операции.

Этот метод включает в себя, в частности, такое понятие, как
конвейеризация. Конвейры широко применяются программистами для решения трудоемких задач, которые можно разделить на этапы, а также в
большинстве современных быстродействующих процессоров \cite{conveyor}.

Целью данной работы является изучение организации конвейерной обработки данных на основе алгоритма раскраски графа по количеству смежных вершин.

В рамках выполнения работы необходимо решить следующие задачи: 
\begin{itemize}[label=---]
	\item изучить основы конвейеризации;
	\item описать алгоритм раскраски графа по количеству смежных вершин;
	\item разработать параллельную версию конвейера для раскраски графа с 3 стадиями обработки;
	\item реализовать линейный и параллельный конвейерный варианты раскраски графа;
	\item провести сравнительный анализ времени работы реализаций.
\end{itemize}