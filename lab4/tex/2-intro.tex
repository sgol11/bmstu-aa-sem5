\specsection{ВВЕДЕНИЕ \hspace{1.25cm}}
\vspace{\baselineskip}

Многопоточность — способность центрального процессора или одного ядра в многоядерном процессоре одновременно выполнять несколько процессов или потоков, соответствующим образом поддерживаемых операционной системой.
Этот подход отличается от многопроцессорности, так как многопоточность процессов и потоков совместно использует ресурсы одного или нескольких ядер: вычислительных блоков, кэш-памяти ЦПУ или буфера перевода с преобразованием.
При последовательной реализации какого-либо алгоритма, его программу выполняет только одно ядро процессора. Если же реализовать алгоритм так, что независимые вычислительные задачи смогут выполнять несколько ядер параллельно, то это приведет к ускорению решения всей задачи в целом.

В тех случаях, когда многопроцессорные системы включают в себя несколько полных блоков обработки, многопоточность направлена на максимизацию использования ресурсов одного ядра, используя параллелизм на уровне потоков, а также на уровне инструкций.
Поскольку эти два метода являются взаимодополняющими, их иногда объединяют в системах с несколькими многопоточными ЦП и в ЦП с несколькими многопоточными ядрами.

Для реализации паралельных вычислений требуется выделить те участки алгоритма,
которые могут выполняться паралльно без изменения итогового результата. 
Также необходимо организовать корректную работу с данными, чтобы не потерять
вычисленные значения.

Целью данной лабораторной работы является получение навыков параллельного программирования на основе алгоритмов численного интегрирования методом средних прямоугольников и методом трапеций.

В рамках выполнения работы необходимо решить следующие задачи:

\begin{itemize}[label=---]
    \item изучить методы средних прямоугольников и трапеций для численного интегрирования;
    \item описать возможности распараллеливания данных алгоритмов;
    \item разработать последовательный и параллельный алгоритмы;
    \item реализовать каждый алгоритм;
    \item провести тестирование реализованных алгоритмов;
    \item провести сравнительный анализ алгоритмов по времени работы
          реализаций.
\end{itemize}