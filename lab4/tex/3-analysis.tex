\section{Аналитический раздел \hfill}
\vspace{\baselineskip}

В данном разделе представлено теоретическое описание алгоритмов численного интегрирования методом средних прямоугольников и методом трапеций.

\numberwithin{equation}{subsection}

\subsection{Метод средних прямоугольников}

Пусть требуется вычислить следующий определенный интеграл:

\begin{equation}\label{eq1}
    I = \int\limits_a^b f(x)dx,
\end{equation}

Предположим, что $f(x)$ непрерывна на $[a,b]$, $n$ -- натуральное и $\Delta x = \frac{b-a}{n}$. Разделим интервал $[a,b]$ на $n$ подынтервалов длиной $\Delta x$ каждый и найдем среднюю точку $m_i$ каждого $i$-ого подыинтервала \cite{integrals}. Тогда определенный интеграл может быть вычислен по следующей формуле:
    
\begin{equation}\label{eq2}
    I_n = \sum\limits_{i=1}^{n} f(m_i) \Delta x,
\end{equation}

\subsection{Метод трапеций}

Пусть требуется вычислить следующий определенный интеграл:

\begin{equation}\label{eq3}
    I = \int\limits_a^b f(x)dx,
\end{equation}

Предположим, что $f(x)$ непрерывна на $[a,b]$, $n$ -- натуральное и $\Delta x = \frac{b-a}{n}$. Разделим интервал $[a,b]$ на $n$ подынтервалов длиной $\Delta x$ каждый. Множество точек, задающих подынтервалы, $P = \{x_0, x_1, \dots, x_n\}$ \cite{integrals}. Тогда определенный интеграл может быть вычислен по следующей формуле:
    
\begin{equation}\label{eq4}
    I_n = \frac{1}{2} \Delta x(f(x_0) + 2f(x_1) + 2f(x_2) + \dots + 2f(x_{n-1}) + f(x_n),
\end{equation}

\subsection{Достижение заданной точности}

Чем больше количество подынтервалов $n$, тем ближе вычисленное значение к реальному значению интеграла, то есть справедлива формула \ref{eq3}.

\begin{equation}\label{eq3}
    I = \lim\limits_{n \to \infty} I_n,
\end{equation}

Понятно, что с технической точки зрения нельзя разделить интервал на
бесконечное число подынтервалов. Это и не требуется, так как необходимой
точности вычислений можно достичь и при конечном $n$. Для этого интервал сначала
разбивают на $m$ и $m + 1$ (начиная с $m=2$) подынтервала, применяют численный
метод для каждого количества и вычисляют разницу между ними. Если разница
меньше заданной точности $\varepsilon$, то вычисления прекращают, а результатом
является последнее вычисленное значение.

\subsection{Параллельные алгоритмы}

В алгоритмах численного интегрирования методом средних прямоугольников и методом трапеций вычисления на каждом из подынтервалов происходят независимо, поэтому есть возможность произвести распараллеливание данных вычислений. Количество отрезков, на которых производит вычисление один поток, будет определяться количеством потоков, а итоговое значение интеграла будет храниться в разделяемой переменной, доступ к которой будут иметь все потоки.

\subsection*{Вывод}

В данном разделе были рассмотрены алгоритмы численного интегрирования методом
средних прямоугольников и методом трапеций. Кроме того, был описан механизм распараллеливания данных алгоритмов.
