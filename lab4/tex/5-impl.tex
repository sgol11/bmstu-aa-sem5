\section{Технологический раздел \hfill}
\vspace{\baselineskip}

В данном разделе описаны требования к программному обеспечению, средства реализации, приведены реализации алгоритмов и данные для тестирования.

\subsection{Требования к программному обеспечению}

Программа должна предоставлять следующие возможности:
\begin{itemize}[label=---]
    \item выбор режима работы (для единичного эксперимента и для массовых эксперименов);
    \item в режиме единичного эксперимента -- выбор функции для интегрирования, ввод пределов интегрирования, точности и числа потоков для параллельной реализации;
    \item в режиме массовых экспериментов -- измерение времени работы каждого из алгоритмов в зависимости от точности и числа потоков.
\end{itemize}

\subsection{Средства реализации}

В качестве языка программирования для реализации лабораторной работы был выбран язык C++ \cite{cpp}. Данный язык предоставляет необходимые библиотеки для работы с потоками. 

Для визуализации данных эксперимента был выбран язык программирования Python \cite{python}, так как он предоставляет большое число настроек параметров графика с использованием простого синтаксиса. 

Для замера процессорного времени использовалась функция библиотеки chrono \cite{chrono} \texttt{std::chrono::system\_clock::now()}.

\subsection{Используемые структуры данных}

В программе используется структура interval\_t, описанная в листинге \ref{lst:interval}. Ее полями являются начало интервала, конец интервала, точность вычислений.

\clearpage

\begin{lstlisting}[label=lst:interval, caption=Структура interval\_t]
struct interval_t
{
    double begin;
    double end;
    double eps;
};
\end{lstlisting}

\subsection{Реализации алгоритмов}
В листингах \ref{lst:calc-int}-\ref{lst:int-trapez} представлены реализации последовательных алгоритмов численного итегрирования методом средних прямоугольников и методом трапеций с заданной точностью. 

В листингах \ref{lst:calc-int-par}-\ref{lst:int-trapez-par} представлены реализации параллельных алгоритмов численного итегрирования методом средних прямоугольников и методом трапеций с заданной точностью. 

\clearpage

\begin{lstlisting}[label=lst:calc-int, caption=Последовательный алгоритм численного интегрирования с заданной точностью]
long double calculate_integral(interval_t &interval, 
            function_t func, long double (&method)(double, 
            double, unsigned int, function_t))
{
    unsigned long n = 2;
    long double delta = 1;
    long double res = 0, obt_res = 0;

    while (delta > interval.eps)
    {
        res = obt_res;
        obt_res = method(interval.begin, interval.end, n, func);
        n *= 2;
        delta = abs(res - obt_res);
    }

    return res;
}
\end{lstlisting}

\clearpage

\begin{lstlisting}[label=lst:int-midpoint, caption=Последовательный алгоритм численного интегрирования методом средних прямоугольников при заданном n]
long double midpoint(double begin, double end, unsigned int n, 
            function_t func)
{
    double res = 0;
    double step = (end - begin) / n;
    double x = begin + step / 2;

    while (x < end + step / 2)
    {
        res += func(x) * step;
        x += step;
    }
    return res;
}
\end{lstlisting}

\begin{lstlisting}[label=lst:int-trapez, caption=Последовательный алгоритм численного интегрирования методом трапеций при заданном n]
long double trapezoidal(double begin, double end, unsigned int n, 
            function_t func)
{
    double res = 0;
    double step = (end - begin) / n;
    double x = begin;

    while (x < end - step)
    {
        res += 0.5 * step * (func(x) + func(x + step));
        x += step;
    }
    return res;
}
\end{lstlisting}

\clearpage

\begin{lstlisting}[label=lst:calc-int-par, caption=Параллельный алгоритм численного интегрирования с заданной точностью]
long double calculate_integral_parallel(int threads_num, interval_t &interval, 
            function_t func, void (&method)(double, double, unsigned int, 
            function_t, long double &, int, int, mutex &))
{
    unsigned long n = 2;
    long double delta = 1;
    long double res = 0, obt_res = 0;

    while (delta > interval.eps)
    {
        res = obt_res;
        obt_res = multithreading(threads_num, interval, n, func, 
                                 method);
        n *= 2;
        delta = abs(res - obt_res);
    }

    return res;
}
\end{lstlisting}

\clearpage

\begin{lstlisting}[label=lst:multithreading, caption=Создание потоков для алгоритмов численного интегрирования]
long double multithreading(int threads_num, interval_t &interval,
            unsigned int n, function_t func, void (&method)
            (double, double, unsigned int, function_t, 
            long double &, int, int, mutex &))
{
    long double res = 0;
    mutex res_mut;
    vector<thread> threads(threads_num);

    for (int i = 0; i < threads_num; i++)
    {
        threads[i] = thread(method, interval.begin, interval.end, 
                            n, func, ref(res), threads_num, i, 
                            ref(res_mut));
    }

    for (int i = 0; i < threads_num; i++)
    {
        threads[i].join();
    }

    return res;
}
\end{lstlisting}

\clearpage

\begin{lstlisting}[label=lst:int-midpoint-par, caption=Параллельный алгоритм численного интегрирования методом средних прямоугольников при заданных n и номере потока]
void midpoint_parallel(double begin, double end, unsigned int n, 
     function_t func, long double &res, int threads_num, int i, 
     mutex &mut)
{
    double local_res = 0;
    double step = (end - begin) / n;
    double x = begin + i * step + step / 2;

    while (x < end + step / 2)
    {
        local_res += func(x) * step;
        x += step * threads_num;
    }

    mut.lock();
    res += local_res;
    mut.unlock();
}
\end{lstlisting}

\clearpage

\begin{lstlisting}[label=lst:int-trapez-par, caption=Параллельный алгоритм численного интегрирования методом трапеций при заданных n и номере потока]
void trapezoidal_parallel(double begin, double end, 
     unsigned int n, function_t func, long double &res, 
     int threads_num, int i, mutex &mut)
{
    double local_res = 0;
    double step = (end - begin) / n;
    double x = begin + i * step;

    while (x < end - step)
    {
        local_res += 0.5 * step * (func(x) + func(x + step));
        x += step * threads_num;
    }

    mut.lock();
    res += local_res;
    mut.unlock();
}
\end{lstlisting}

\subsection{Тестирование}

В таблице \ref{tab:tests} приведены функциональные тесты для алгоритмов
интегрирования на функции $f(x) = x^2$. Тесты пройдены успешно.

\begin{table}[h!]
	\begin{center}
    \begin{threeparttable}
        \captionsetup{justification=raggedright, singlelinecheck=off}
        \caption{\label{tab:tests}Функциональные тесты}
        \begin{tabular}{|c|c|}
			\hline
            \textbf{Пределы интегрирования} & \textbf{Ожидаемый результат} \\ [2mm]
            \hline
            0 0
            &
            0
            \\
            \hline
            0 1
            &
            0.3333
            \\
            \hline
            1 0
            &
            -0.3333
            \\
            \hline
            -1 1
            &
            0.6666
            \\
            \hline
		\end{tabular}
    \end{threeparttable} 
	\end{center}
\end{table}

\subsection*{Вывод}

Были описаны требования к программному обеспечению, средства реализации, приведены реализации алгоритмов и тесты, успешно пройденные программой.
